\documentclass[12pt]{scrartcl}

\usepackage[yyyymmdd,hhmmss]{datetime}

\usepackage{epsfig,amssymb}

\usepackage{xcolor}
\usepackage{graphicx}
\usepackage{epstopdf}
\usepackage{multirow}

\definecolor{darkred}{rgb}{0.5,0,0}
\definecolor{darkgreen}{rgb}{0,0.5,0}
\definecolor{steelblue}{RGB}{70,130,180}
\definecolor{darksteelblue}{RGB}{56,104,144}
\usepackage[pdfusetitle]{hyperref}
\hypersetup{
  letterpaper,
  colorlinks,
  linkcolor=red,
  citecolor=darkgreen,
  menucolor=darkred,
  urlcolor=darksteelblue,
  pdfpagemode=none,
}

\usepackage{fullpage}
\pagestyle{empty} %

\definecolor{mintedBackground}{rgb}{0.95,0.95,0.95}
\definecolor{mintedInlineBackground}{rgb}{.90,.90,1}

\usepackage[newfloat=true]{minted}

\setminted{mathescape,
           linenos,
           autogobble,
           frame=none,
           framesep=2mm,
           framerule=0.4pt,
           xleftmargin=2em,
           xrightmargin=0em,
           numbersep=10pt, %gap between line numbers and start of line
           style=default} %syntax highlighting style, default is "default"

\setmintedinline{bgcolor={mintedBackground}}
%doesn't work with the above workaround:
\setminted{bgcolor={mintedBackground}}
\setminted[text]{bgcolor={mintedBackground},linenos=false,autogobble,xleftmargin=1em}
%\setminted[php]{bgcolor=mintedBackgroundPHP} %startinline=True}

\setlength{\parindent}{0pt} %
\setlength{\parskip}{.25cm}
\newcommand{\comment}[1]{}

\usepackage{lastpage}

\usepackage{fancyhdr}
\renewcommand*{\titlepagestyle}{fancy}
\pagestyle{fancy}
\renewcommand{\headrulewidth}{0.0pt}
\renewcommand{\footrulewidth}{0.4pt}
\lfoot{\Title}
\cfoot{~}
\rfoot{\thepage\ / \pageref*{LastPage}}

%remove headers
\rhead{~}
\lhead{~}


\makeatletter
\title{CSE Faculty Learning Community Handbook}\let\Title\@title
\subtitle{~\\
{\small
\vskip1cm
Department of Computer Science \& Engineering \\
University of Nebraska--Lincoln}
\vskip-1cm}
\date{\small\today\  \currenttime \\ Version 0.1.0}
\makeatother


\begin{document}

\maketitle

\hrule

\section{Introduction}

This document provides a list of resources including 
policies, procedures and other items related to teaching 
for faculty in the Department of Computer Science \& 
Engineering at the University of Nebraska--Lincoln.

\subsection{Related Documents}

The Department of Computer Science \& Engineering maintains
two other important documents that faculty should be familiar
with.  Both of these documents are available on the secure
faculty webpage: \url{https://cse.unl.edu/faculty/protected/} 
which requires a cse login.

\begin{itemize}
  \item Department Bylaws (revised 2016/10/13)
  \item Faculty Personnel Policies (revised 2016/10/13)
\end{itemize}

\section{Resources}

\subsection{System Resources}

The CSE Systems Staff maintains a systems-related FAQ for
faculty, staff and students available here:
\url{https://cse.unl.edu/faq} most systems-related questions
and issues can be answered here.  If you have an issue that
is not addressed by this FAQ or have any other systems-related
questions, you can send an email to \href{mailto:manager@cse.unl.edu}{manager@cse.unl.edu}.

\subsection{Accounts}

There are several systems and accounts that you'll have access
to:

\begin{itemize}
  \item CSE Account -- The main CSE server, cse.unl.edu is managed
  by the systems administration staff.  Account management can be
  handled through the web interface:  \url{https://cse-apps.unl.edu/amu/amu/login}
  \item Canvas -- Canvas is UNL's Learning Management System (LMS) 
  and available directly at \url{https://canvas.unl.edu/}.  Use
  your ``MyUNL'' login to access the LMS.  This will be your primary
  tool as an instructor.  You typically use Canvas for grades, to
  deliver course content and communication with students.
  \item MyRed -- \url{https://myred.nebraska.edu} is typically used
  for rosters and final grades but there is a lot more data that
  is accessible in this system.  Visit with a department secretary 
  for getting additional access.
  \item Firefly -- \url{https://firefly.nebraska.edu/} is used for
  HR-related items including pay check inqueries, tax information, 
  benefits, etc.  It is also used to ``Approve Attendance'' 
  (approving work hours) of Undergraduate Teaching Assistants 
  that have been assigned to you.
  \item Course Accounts: for each course you teach you can optionally
  request a course account (example: for CSCE 310 the course account
  login would be \mintinline{text}{cse310}).  Course accounts are
  necessary to use the cse webhandin and webgrader system.  They also
  come with their own print quota.  You must request access each
  semester for each course by sending an email to 
  \href{mailto:manager@cse.unl.edu}{manager@cse.unl.edu} with the
  relevant information.  Course accounts expire and are
  archived at the end of each semester.
\end{itemize}

\subsection{Software \& Technology Resources}

Information Technology Services (ITS, \url{https://its.unl.edu/}) 
offers a wide array of software solutions and services.  A
comprehensive catalog of available services is available here:
\url{https://its.unl.edu/services/}

\subsection{Website Hosting}

There are a couple of options for hosting websites as a 
faculty member.

\begin{itemize}
  \item Create a \mintinline{text}{public_html} directory
  and include web content in it which is then accessible via the
  url \url{https://cse.unl.edu/~login} (where \mintinline{text}{login}
  is replaced with your cse login).  This is a typical apache server
  that supports CGI, htaccess, etc.
  \item You can request your own site hosted by ITS
  \url{https://its.unl.edu/services/unlcms/} which is a Drupal
  CMS (Content Management System) that automatically 
  integrates UNL's web templates.  Custom URLs can also
  be requested (example: gracehopper.unl.edu).
\end{itemize}

\subsection{CSE Faculty Resources}

A secure faculty page is maintained by the systems administrators
at \url{http://cse.unl.edu/faculty/protected}.  In this page you can:
\begin{itemize}
  \item Make room reservations for CSE-controlled rooms in Avery and Schorr
  \item Access department documents including course specifications
  \item Access the faculty database and update your profile/internal CV
  \item Arrange a course (internships, etc.)
  \item Access the department course syllabus policy and required text
\end{itemize}

\subsection{ECEC}

The Engineering and Computing Education Core (ECEC, \url{https://engineering.unl.edu/ecec})
is a facility in the College of Engineering providing professional development
opportunities and resources on teaching.  Some of their items:
\begin{itemize}
  \item Teaching Reflections including Peer Evaluation reflections: \url{https://engineering.unl.edu/ecec/resources-faculty/reflecting/}
  \item Syllabus Template: \url{https://engineering.unl.edu/ecec/ecec-fall-teaching-templates/}
\end{itemize}

\subsection{Webhandin \& Webgrader}

CSE maintains a webhandin system that allows students to submit
files electronically through a web interface: \url{https://cse-apps.unl.edu/handin/} (use your cse login).  As an instructor you
can:
\begin{itemize}
  \item Setup assignments and their due dates (with separate late
and ``close'' dates after which students will not be able to
submit or alter files)
  \item Restrict file names/types using regular expressions
  \item Add/remove users (students, graders, etc.)
  \item View handin logs (it keeps the last 5 versions of
  each file handed in)
  \item Download individual files or archives of all submissions
\end{itemize}

A companion system, the cse webgrader enables you to expose
automated grading scripts to students using their cse login so
that they can grade/verify their own submissions.  The system
itself only consists of the web interface, you still need to
specify the assignment details and write your own scripts/test
cases, but example scripts (PHP) are provided.

\begin{itemize}
  \item You can clone the project via GitHub, in your course
  account's \mintinline{text}{public_html}: 
   \mintinline{text}{git clone https://github.com/cbourke/grade}
  \item The repo (\url{https://github.com/cbourke/grade}) has a
  readme with more details
  \item We have produced an introductory video for instructors 
  and teaching assistants on setting up and using the system:
  \url{https://www.youtube.com/watch?v=CRvXsOfp1Vo}
\end{itemize}

\subsection{Digital Learning Center (Testing Center)}

UNL has a Digital Learning Center (\url{https://its.unl.edu/dlc/})
that provides several services including scantron services and 
web-based exam services.  You can develop online exams through
Canvas using Mobius (MapleTA) and schedule time(s) for students
to take the exam in the DLC.  

\subsection{UNL Writing Center}

Students can be referred to the UNL Writing Center (\url{https://www.unl.edu/writing/home}) for help with their writing including
assignments, developing resumes/cover letters, etc.  The Writing
Center also provides services to faculty if you need help with
your writing or want to collaborate with them on a writing assignment
or project for your class.

\subsection{MyPLAN}

MyPLAN is an app hosted within Canvas that allows you to
track students in your courses.  This app enables you to
raise flags (missing assignments, attendance issues, performance
issues) which may result in advisor or first year experiences
(\url{https://success.unl.edu/}) intervention.  You can also
raise positive ``Kudos'' for your students.

\subsection{Advising}

CSE has several embedded professional advisors that you can
refer students to for advising services.  Advising resources
can be found at \url{http://cse.unl.edu/advising} 

\subsection{Office Supplies \& Materials}

You can get office supplies as well as textbook copies from the
department staff.  See Matt Wagenheim for your needs.

\subsection{GTAs, UTAs, LAs}

The department has Graduate Teaching Assistants (GTAs), 
Undergraduate Teaching Assistants and Learning Assistants
to help you with your courses.  You will receive an assignment
each semester and may request more/fewer TAs by talking
with the relevant faculty (see below).

\subsection{Communication}

You can send a message to all CSE faculty by emailing 
\href{mailto:faculty@cse.unl.edu}{faculty@cse.unl.edu}. 

UNL maintains a listserv server where you can create and
maintain listserv mailing lists: \url{https://listserv.unl.edu/}

Several listservs and news feeds that may be of interest:
\begin{itemize}
  \item Teacher Connect: \url{https://newsroom.unl.edu/announce/teacherconnect}
  \item \mintinline{text}{TEACHING-AND-TECH} listserv (join at \url{https://listserv.unl.edu/})
  \item \mintinline{text}{DBER_FACULTY} listserv (join at \url{https://listserv.unl.edu/})
\end{itemize}

\subsection{Computing Education Research}

As part of professional development, faculty can keep up
on the latest computing education research as published in 
the two major CS-Ed conferences:
\begin{itemize}
  \item SIGCSE (\url{https://sigcse.org/sigcse/}), ACM's Special Interest Group on Computer Science Education (which also has an excellent listserv community, but ACM membership is required to join)
  \item ITiCSE (\url{https://sigcse.org/sigcse/events/iticse}), SIGCSE's Innovation and Technology in Computer Science Education conference
  \item CCSC (\url{https://www.ccsc.org/midwest/}) is the Consortium for Computing Sciences in Colleges (we're in the midwest region)
  \item ASEE (\url{https://www.asee.org/}) is the American Society for Enginneering Education which is more broadly focused on all Engineering disciplines.  
\end{itemize}

Locally, Nebraska (mostly Lincoln and Omaha) has a Chapter of the 
Computer Science Teachers Association (NE CSTA, 
\url{http://cstanebraska.org/}) where you can network with local
K-12 CS-oriented teachers.  

\subsection{Professional Development Opportunities}

\begin{itemize}
  \item CoE's Engineering and Computing Education Core (ECEC,
  \url{https://engineering.unl.edu/ecec/})   
  offers a variety of professional development programs for
  faculty: \url{https://engineering.unl.edu/ecec/faculty-programs/}
  \item The Discipline-Based Eduction Research group (DBER, \url{http://scimath.unl.edu/dber/}) also offers a variety of professional
  development programs: \url{http://scimath.unl.edu/dber/professional-development/}
  \item Senior Vice Chancellor for Academic Affairs offers a 
  Teaching \& Learning Symposium once each semester with numerous
  breakout sessions: \url{https://executivevc.unl.edu/faculty/teaching-learning-symposium}
  \item The Innovation in Pedagogy and Technology Symposium is an 
  annual event held in May: \url{https://symposium.nebraska.edu/}
  \item The Century Club (usually held 3:30 on the 2nd Tuesday of
  each month) is a monthly symposium on various technologies and
  techniques for larger classroom settings
  \item UNL has established a Teaching at UNL Canvas ``course''
  which houses several teaching-related resources: \url{https://canvas.unl.edu/courses/51131}
\end{itemize}

In addition, you can get involved with the Faculty Senate's
Teaching Council (\url{https://www.unl.edu/facultysenate/teaching-council})

\section{Policies}

\subsection{Syllabi}

\begin{itemize}
  \item UNL has a general syllabus policy: \url{https://www.unl.edu/facultysenate/unl-syllabus-policy}
  \item The College of Engineering does not have an official syllabus \emph{policy} but ECEC (Engineering and Computing Education Core) does have a suggested template that you are free to adopt and adapt: \url{https://engineering.unl.edu/ecec/ecec-fall-teaching-templates/} (as well as a Canvas course template and other resources)
  \item The department has an additional syllabus policy with required text to be included: 
\url{https://cse.unl.edu/faculty/protected/SyllabusText.html} (faculty login required) 
\end{itemize}
 
\subsection{Department Academic Integrity Policy}

UNL has a broad Academic Integrity policy as part of the
student code of conduct (\url{https://studentconduct.unl.edu/academic-integrity}).  The department also maintains its own Integrity
Policy available here: \url{https://cse.unl.edu/academic-integrity-policy}.

Though these policies are in place, as an instructor it is up
to you to define what constitutes a violation of this policy.
Your own course-specific policies should be well thought out and
clearly stated in your syllabus.  Violations of the integrity
policy and actions taken against a student must be reported to
the department's Academic Integrity \& Grading Appeals Committee.

\subsection{15th Week Policy}

UNL has a 15th week policy colloquially known as the ``Dead Week
Policy'' (see \url{https://studentaffairs.unl.edu/sa_policies_deadweek.shtml}).  This policy is intended to prevent
``surprise'' assignments or exams at the end of the semester.
If you want to assign any substantial assignments (homework, 
exams, etc.) you need to do so in writing prior to the 8th week
of classes.  

It is best to have your course as well as all assignments and
due dates completely planned out at the beginning of the semester
and make this schedule known to students to avoid violating this
policy.  

\subsection{Family Educational Rights and Privacy Act (FERPA)}

The Family Educational Rights and Privacy Act is a federal law
that prevents faculty and staff from violating a student's privacy
with respect to any academic record(s) (grades, flags, contact
information, addresses, etc.) that you have access to.  

You should never make any identifying information publicly available
or discuss the performance of a student with anyone who is 
not bound by this law (other faculty/staff employed by UNL are
bound by this law) unless the student has signed a FERPA waiver.

\subsection{Counseling and Psychological Services (CAPS)}

UNL provides counseling services through CAPS 
(\url{https://caps.unl.edu/}) for students (free unlimited 
sessions).  If you believe a student needs such services, please
refer them to CAPS.

\subsection{Services for Students with Disabilities (SSD)}

Students with a documented learning disability or other impairment
that requires accommodations should be registered with the 
SSD office (\url{https://www.unl.edu/ssd/home}).  SSD will handle
deciding the appropriate accommodations and working directly
with a student and their doctor to protect their privacy.  
It is a student's responsibility to provide the proper paperwork
and you can choose to have the SSD office provide the appropriate
accommodations (proctoring in a distraction free environment, 
extended time on exams, etc.).

\subsection{Greeks, Honors, Scholarships, Student Athletes, ROTC, etc.}

Other student groups will often ask for accommodations and/or
grade checks.  Making accommodations are generally not required
for these groups but you should make reasonable efforts to do so
in order to support the student.  

UNL Honors students may want to ``contract'' your course for honors
credit.  Generally, the extra requirements to fulfill this are up
to you and the student.


\section{Whom to Talk to About...}

Current faculty committee chairs and member lists can be found
here: \url{https://cse.unl.edu/faculty-committees}

\begin{itemize}
  \item Graduate Teaching Assistants (GTAs): Mohammad Hasan (GTA Committee Chair)
  \item Undergraduate Teaching Assistants (UTAs): Ryan Patrick
  \item Learning Assistants (LAs): Ashok Samal
  \item Academic Integrity: Chris Bohn (Academic Integrity \& Grading Appeals Committee)
  \item ABET Accreditation: Chris Bohn
  \item Internships and Arranged Courses: Ryan Patrick
  \item Course Scheduling \& Logistics: Department Vice Chair
  \item Office Supplies: Matt Wagenheim
  \item Textbook copies: Matt Wagenheim
\end{itemize}

\section{Teaching Evaluation}

To Come

\end{document}